\documentclass[nobib]{tufte-handout}

\title{Övningstillfälle 1: Programmeringsövningar $\cdot$ 1MA020}

\author[Vilhelm Agdur]{Vilhelm Agdur\thanks{\href{mailto:vilhelm.agdur@math.uu.se}{\nolinkurl{vilhelm.agdur@math.uu.se}}}}

\date{23 januari 2023}


%\geometry{showframe} % display margins for debugging page layout

\usepackage{graphicx} % allow embedded images
  \setkeys{Gin}{width=\linewidth,totalheight=\textheight,keepaspectratio}
  \graphicspath{{graphics/}} % set of paths to search for images
\usepackage{amsmath}  % extended mathematics
\usepackage{booktabs} % book-quality tables
\usepackage{units}    % non-stacked fractions and better unit spacing
\usepackage{multicol} % multiple column layout facilities
\usepackage{lipsum}   % filler text
\usepackage{fancyvrb} % extended verbatim environments
  \fvset{fontsize=\normalsize}% default font size for fancy-verbatim environments

\usepackage{color,soul} % Highlights for text


\include{mathcommands.extratex}

\begin{document}

\maketitle% this prints the handout title, author, and date

\begin{abstract}
\noindent
Detta dokument innehåller en samling övningar i kombinatorik som kräver programmering: Vi utforskar alltså några olika saker med hjälp av stora mängder beräkningar.
\end{abstract}

Mycket av det vi diskuterat hittills i kursen har varit att räkna olika sätt att fördela särskiljbara eller osärskiljbara objekt mellan lådor, som likaledes kan vara särskiljbara eller osärskiljbara. I dessa övningar utforskar vi lite mer komplicerade varianter av samma sorts problem, där det vore för invecklat eller avancerat att bevisa resultat om dem under föreläsningarna, men vi ändå kan greppa problemet numeriskt.

En sak som kommer återkomma senare i kursen, men som lär kunna vara mycket användbart redan nu, är kopplingen mellan sannolikheter och antal. Om vi tänker oss att vi har en urna full med $n$ stycken röda och blåa bollar, och vi vet att sannolikheten att få en röd boll om vi tar en slumpmässig boll är $p$, vet vi så klart också att \emph{antalet} röda bollar är $np$.

Vi kan vända på det här resonemanget: Om $n$ är väldigt stort kommer det vara svårt att räkna precis hur många röda bollar det finns, eftersom vi behöver kolla på varje boll, men vi kan få en god skattning av ungefär hur många röda bollar det finns genom att helt enkelt slumpmässigt välja bollar och räkna ut vilken andel av de slumpmässigt valda bollarna som är röda. Den andelen kommer vara nära den sanna sannolikheten att få en röd boll, och den sannolikheten bestämmer i sin tur exakt hur många röda bollar det finns.

När vi vill räkna saker på datorn får vi ibland använda samma princip: Det är ofta så att det finns alldeles för många objekt för att vi skall kunna representera allihop i datorns minne, eller att det vore alldeles för komplicerat att verkligen enumerera alla, men däremot är relativt enkelt att generera ett slumpmässigt objekt.\sidenote[][]{Notera dock att vi, för att detta skall fungera, måste generera varje objekt med samma sannolikhet: Om en del bollar är sannolikare att plockas upp än andra så fallerar hela vår metod. Så fundera över om er metod för att välja objekten verkligen ger samma sannolikhet till varje objekt.}

\section{Bollar i urnor}

Vi fortsätter på temat med bollar i urnor,\sidenote[][]{Det är stereotypt av en anledning...} genom att studera en så kallad Pólya-urna. Modellen är som följer: Vi har en urna som innehåller $n_0$ röda bollar och $m_0$ blåa bollar. Sedan plockar vi upp en boll slumpmässigt: Om den är röd lägger vi tillbaka $r_r$ röda bollar och $r_b$ blåa bollar i urnan, och om den är blå lägger vi tillbaka $b_r$ röda bollar och $b_b$ blåa bollar. Vi kallar det nya antalet röda och blåa bollar $n_1$ och $m_1$, och fortsätter processen.

Det kan vara nyttigt att formulera modellen i ett mer linjäralgebraiskt språk: Vi börjar med en vektor $[n_0, m_0]^T$ som vår initiala konfiguration, och sedan bestäms hur processen utvecklas av en \emph{övergångsmatris}
$$\begin{bmatrix}
    r_r & r_b \\
    b_r & b_b 
\end{bmatrix}.$$

\begin{xca}
    Skriv kod för att simulera denna processen, med parametrar enligt vår beskrivning ovan.
\end{xca}

När vi nu har koden för att göra det hela i allmänhet, låt oss studera ett par enkla exempel, och fundera på hur vi tror de kommer att bete sig.

\begin{xca}
    För varje av de nedanstående exemplen, fundera först utan att använda er kod på hur ni tror att svaret kommer att bli, och skriv ner ert resonemang. Gör sedan simuleringen och generera ett par lämpliga grafer för att illustrera vad som faktiskt hände. Stämde ert resonemang, eller var resultatet oväntat?

    \begin{enumerate}
        \item Vi börjar med $[n_0, m_0]^T = [1, 1]^T$, och har övergångsmatris
        $$\begin{bmatrix}
            2 & 0 \\
            0 & 2 
        \end{bmatrix}.$$

        En lämplig figur att rita här kan vara hur andelen röda bollar -- alltså $\frac{n_t}{n_t + m_t}$ -- utvecklar sig över tiden, för olika realisationer av processen.

        \item Vi börjar med något $[n_0, m_0]^T$, och har övergångsmatris
        $$\begin{bmatrix}
            1 & 2 \\
            2 & 1 
        \end{bmatrix}.$$

        Kommer utvecklingen av andelen röda bollar alltid att se likadan ut om vi börjar med samma $n_0$ och $m_0$? Hur ser det ut om vi varierar våra startförutsättningar, med olika andelar röda bollar till en början?

        \item Vi börjar med $[n_0, m_0]^T = [0, m]^T$, och har övergångsmatris
        $$\begin{bmatrix}
            1 & 0 \\
            1 & 0 
        \end{bmatrix}.$$

        Nu har vi ju plötsligt en asymmetri mellan bollarna -- plockar vi upp en röd boll lägger vi bara tillbaka den, men plockar vi upp en blå boll målar vi den röd innan vi lägger tillbaka den. Hur många gånger kommer vi behöva plocka upp en boll innan det bara finns röda bollar kvar?

        \item I vårt sista exempel vill vi jämföra två olika urnor. I bägge börjar vi med $[n_0, m_0]^T = [1,1]^T$. De två urnorna har övergångsmatriser
        $$\begin{bmatrix}
            6 & 1 \\
            0 & 2 
        \end{bmatrix}\qquad\text{respektive}\qquad\begin{bmatrix}
            6 & 1 \\
            0 & 1 
        \end{bmatrix}.$$

        Så de två fungerar nästan precis likadant, förutom att i den första ökar antalet blåa bollar med ett när vi plockar upp en blå boll, och i den andra lägger vi bara tillbaka den blå bollen igen. Frågan är denna: Kommer det vara en signifikant skillnad i hur snabbt antalet blåa bollar växer i de två modellerna?
    \end{enumerate}
\end{xca}

\section{Lappar i urnor}

Låt oss nu istället studera en modell i vilken vi har papperslappar med streck på i en urna.\sidenote[][]{Om man vill kan man tänka på det här som en Pólya-urna med oändligt många möjliga färger på bollarna, men det gör det nog inte enklare att förstå om man inte redan är väldigt bekväm med Pólya-urnor.} Vi börjar med en enda tom lapp. I varje tidssteg plockar vi upp en slumpmässig lapp, ritar ett till streck på den, och lägger tillbaka lappen och en ny tom lapp i urnan.

\begin{xca}
    Skriv kod som simulerar den här processen. Hur utvecklas värdet av talet som står på den största lappen med tiden? Rita ett histogram över värdena som står på lapparna.\sidenote[][]{Fundera gärna på vad ni tror svaret kommer att bli innan ni faktiskt genomför uppgiften. Hur snabbt tror ni det här värdet kommer att växa? Om värdet på den största lappen är $\alpha_n$ vid tid $n$, är $\alpha_n$ ungefär $0.6 n$? Ungefär $n^2$? Ungefär $\log(n)$?
    
    Hur kommer histogrammet se ut? Vad kommer vara det vanligaste, näst vanligaste, och så vidare, talet att se på en lapp?}
\end{xca}

\section{Matematiker i bussar}

Antag att vi har $n$ stycken matematiker som skall kliva på $k$ stycken bussar för att resa till en konferens i Budapest. Hur många sätt finns det att placera ut matematikerna på bussar? Hur många sätt finns det så att åtminstone en av bussarna bara innehåller en matematiker?

\begin{xca}
    Skriv kod för att räkna ut dessa kvantiteter. Dels med exakta beräkningar för små värden på $n$ och $k$, och dels genom simuleringar av slumpmässiga placeringar för stora värden.
\end{xca}

\begin{xca}
    Tänk att vi håller antalet matematiker, $n$, fixt, medan vi ökar antalet bussar, $k$. Kommer sannolikheten att någon av bussarna innehåller en ensam matematiker att öka med $k$?\sidenote[][]{Upgiften är alltså att studera detta med hjälp av er kod, inte teoretiskt. Men om ni lyckas hitta ett korrekt bevis för att denna sannolikhet faktiskt är ökande vore detta mycket intressant. (Om ni lämnar in ett sådant får ni definitivt full pott på uppgiften, alldeles oavsett resten av uppgiften.)} Vad tror ni intuitivt, och vad säger era simuleringar?
\end{xca}


%\bibliography{references}
%\bibliographystyle{plainnat}

\end{document}
