\documentclass[nobib]{tufte-handout}

\title{Föreläsning 5: Genererande funktioner $\cdot$ 1MA020}

\author[Vilhelm Agdur]{Vilhelm Agdur\thanks{\href{mailto:vilhelm.agdur@math.uu.se}{\nolinkurl{vilhelm.agdur@math.uu.se}}}}

\date{31 januari 2023}


%\geometry{showframe} % display margins for debugging page layout

\usepackage{graphicx} % allow embedded images
  \setkeys{Gin}{width=\linewidth,totalheight=\textheight,keepaspectratio}
  \graphicspath{{graphics/}} % set of paths to search for images
\usepackage{amsmath}  % extended mathematics
\usepackage{booktabs} % book-quality tables
\usepackage{units}    % non-stacked fractions and better unit spacing
\usepackage{multicol} % multiple column layout facilities
\usepackage{lipsum}   % filler text
\usepackage{fancyvrb} % extended verbatim environments
  \fvset{fontsize=\normalsize}% default font size for fancy-verbatim environments

\usepackage{color,soul} % Highlights for text


\include{mathcommands.extratex}

\begin{document}

\definecolor{darkgreen}{rgb}{0.0627, 0.4588, 0.1451}

\maketitle% this prints the handout title, author, and date

\begin{abstract}
\noindent
Vi går vidare från våra grundläggande tekniker till en lite mer avancerad metod inom kombinatoriken -- genererande funktioner.
\end{abstract}

Hittills har vi bevisat våra resultat huvudsakligen med hjälp av smarta insikter i strukturen hos problemen -- med kombinatoriska argument som ser på samma objekt ur två vinklar, eller ser en bijektion. Det enda större verktyget vi introducerat hittills är inklusion-exklusion.

Det är dags att introducera ett mer systematiskt verktyg som kan användas för många olika problem, och som låter oss använda våra färdigheter i algebra för att uttrycka och lösa kombinatoriska problem.

\section{Genererande funktioner}

\begin{definition}
    Antag att vi har en talföljd $a_0, a_1, a_2, \ldots$. Beteckna denna som $\{a_k\}_{k=0}^\infty$. Den \emph{genererande funktionen} för denna talföljd ges av
    $$F_{a}(x) = \sum_{k=0}^{\infty} a_k x^k.$$
\end{definition}

I den här kursen betraktar vi dessa funktioner som helt och hållet \emph{kombinatoriska} objekt -- vi bryr oss inte ett dugg om ifall dessa uttryck faktiskt konvergerar eller inte.\sidenote[][]{Det finns intressanta tillämpningar av att räkna ut konvergensradien för dessa uttryck -- det kan säga oss något om hur stort $a_k$ är asymptotiskt, alltså för väldigt stora $k$. Men det är överkurs för oss.} Vi bryr oss för det mesta inte ens om att evaluera dem i någon punkt. De är helt och hållet formella objekt som vi bara manipulerar enligt algebrans räkneregler.\sidenote[][]{Det här går att göra rigoröst med en bunt algebra-hokuspokus och termer som ``polynomring i oändligt många variabler''. Vi skippar det.}

\begin{example}
    Välj ett fixt heltal $n$, och låt $a_k = \binom{n}{k}$ för varje $k$.\sidenote[][]{Specifikt blir alltså $a_k = 0$ för $k > n$, eftersom en mängd har noll delmängder större än sig själv.} Den genererande funktionen för denna följd blir då
    $$F_a(x) = \sum_{k=0}^{n}\binom{n}{k}x^k.$$

    Om vi använder binomialsatsen kan vi få ett enklare uttryck för denna genererande funktion, eftersom den ger oss att
    $$(1+x)^n = \sum_{k= 0}^{n}\binom{n}{k}1^{n-k}x^k = F_a(x).$$
\end{example}

\begin{example}
    Välj ett fixt heltal $n$, och låt följden $a_k$ ges av att $a_k = 1$ om $k \leq n$ och $0$ annars. Dess genererande funktion ges då av
    $$\sum_{k=0}^{n} x^k = 1 + x + \ldots + x^n.$$

    Vi kan få den på en enklare form genom att observera att
    \begin{align*}
        1 - x^{n+1} &= (1 + x + x^2 + \ldots + x^n) - (x  + x^2 + \ldots + x^{n+1})\\
        &= (1 - x)(1 + x + x^2 + \ldots + x^n) = (1 - x)F_a(x)
    \end{align*}
    och lösa detta uttryck för $F_a(x)$ och få
    $$F_a(x) = \frac{1 - x^{n+1}}{1 - x}.$$
\end{example}

\begin{example}
    Låt $a_k = 1$ för alla $k$. Dess genererande funktion är då
    $$F_a(x) = \sum_{k=0}^{\infty} x^k$$
    vilket vi kan känna igen som en oändlig geometrisk summa, för vilka vi vet att formeln är\sidenote[][]{Eller så hade vi kunnat använda samma räkning som i förra exemplet, även om den är lite svårare att rättfärdiga. Eller så känner vi igen det som Taylorserien för $\frac{1}{1-x}$.}
    $$F_a(x) = \frac{1}{1-x}.$$
\end{example}

\section{Rekursioner}

En fråga ni bör ställa er i det här läget är denna: Vad är allt det här bra för? Vi har tagit följder, definierat serier för dem, och hittat uttryck för serierna. Än sen då?

Nyttan med genererande funktioner är till stor del att information från den ``kombinatoriska'' sidan återspeglas i de genererande funktionerna -- så vi kan ta information på ena sidan, manipulera på den sidan, och sedan gå tillbaka till andra sidan och ha lärt oss något nytt. Ofta går vi i riktningen kombinatorik till genererande funktion till kombinatorik, eftersom vi har så mycket mer kunskap om hur man resonerar om funktioner och algebra.

Som ett första exempel på detta, låt oss använda genererande funktioner för att studera rekursioner och rekurrensrelationer.

\begin{example}
    Låt oss studera Fibonacciföljden $\{f_k\}_{k=0}^\infty$, som ges av att $f_0 = f_1 = 1$ och
    $$f_{k+1} = f_k + f_{k-1}$$
    för alla $k \geq 1$.

    Vad är dess genererande funktion? Vi tar vår rekursion för den och multiplicerar med $x^k$, och får att
    $$f_{k+1} x^k = f_k x^k + f_{k-1} x^k$$
    så om vi summerar detta över alla $k \geq 1$ (eftersom dessa är de $k$ för vilka likheten är giltig) får vi att
    $$\sum_{k=1}^{\infty} f_{k+1}x^k = \sum_{k=1}^{\infty} f_k x^k + \sum_{k=1}^{\infty} f_{k-1}x^k.$$

    Vi ser att alla uttrycken här ser väldigt snarlika ut genererande funktionen för Fibonacciföljden, men ingen av dem är exakt den genererande funktionen. Så om vi manipulerar uttrycken lite får vi att
    $$\frac{1}{x}\sum_{k=1}^{\infty} f_{k+1}x^{k+1} = \sum_{k=1}^{\infty} f_k x^k + x\sum_{k=1}^{\infty} f_{k-1}x^{k-1}$$
    vilket är ännu närmre. Sista termen är nu precis den genererande funktionen, men de andra startar summan för högt -- de skippar de första termerna. Så om vi justerar för detta genom att lägga till noll på ett par ställen får vi att
    $$\frac{1}{x}\left(\sum_{k=1}^{\infty} f_{k+1}x^{k+1} + (f_0 - f_0 + f_1x - f_1x)\right) = \left(\sum_{k=1}^{\infty} f_k x^k + (f_0 - f_0)\right) + xF_f(x)$$
    och nu, när vi flyttar in de extra $f_0$ och $f_1x$ vi har köpt oss är uttrycken faktiskt precis den genererande funktionen, och vad vi har är att
    $$\frac{1}{x}\left(F_f(x) - f_0 - f_1x\right) = F_f(x) - f_0 + xF_f(x)$$
    vilket, om vi kommer ihåg våra initialförutsättningar att $f_0 = f_1 = 1$, blir till att
    $$\frac{F_f(x) - x - 1}{x} = F_f(x) - 1 + xF_f(x).$$

    Vi har alltså omvandlat det kombinatoriska påståendet om vår rekursion till ett algebraiskt påstående om vår genererande funktion. Och till skillnad från rekursionen kan vi ju enkelt lösa ekvationen för vår genererande funktion och få att
    $$F_f(x) = \frac{1}{1 - x^2 - x}.$$
\end{example}

Det här tar oss alltså halvvägs till att ha gjort något intressant -- vi har gått från kombinatorik till genererande funktion, och manipulerat den kombinatoriska informationen algebraiskt för att få fram ny information om den genererande funktionen. Men vi är ju intresserade av den kombinatoriska sidan, så vi vill ju översätta den här informationen tillbaka till att säga något intressant om Fibonacciföljden.

En sak man kan göra, men som vi inte skall lägga tid på\sidenote[][]{Vad man får göra är ofta att partialbråksuppdela den genererande funktionen, och att partialbråksuppdela är ju ungefär det tråkigaste man kan göra i matematiken, alltså varför vi skippar att faktiskt göra det.} i denna föreläsningen, är att räkna ut Taylorutvecklingen av denna genererande funktionen, och på så vis få fram en formel för det $n$te Fibonaccitalet.

Ifall man faktiskt genomför den räkningen kommer man att få följande resultat:

\begin{proposition}[Binets formel]
    Det gäller för Fibonaccitalen att
    $$f_n = \frac{\phi^n - \phi^{-n}}{\phi - \frac{1}{\phi}}$$
    där $\phi = \frac{1 + \sqrt{5}}{2}$ är det gyllene snittet.
\end{proposition}

\begin{lemma}[Räkneregler för genererande funktioner]\label{lemma_generating_function_calc_rules}
    Antag att vi har en följd $\{a_k\}_{k=0}^\infty$, med genererande funktion $F_a$. Då gäller det att
    \begin{enumerate}
        \item För varje $j \geq 1$ är
        $$\sum_{k = j}^{\infty} a_k x^k = \left(\sum_{k=0}^{\infty}a_k x^k\right) - \left(\sum_{k=0}^{k=j-1} a_kx^k\right) = F_a(x) - \sum_{k=0}^{k=j-1} a_kx^k$$
        \item För alla $m \geq 0$, $l \geq -m$ gäller det att
        $$\sum_{k=m}^{\infty} a_k x^{k + l} = x^l\left(\sum_{k=m}^{\infty} a_k x^{k}\right) = x^l\left(F_a(x) - \sum_{k=0}^{m-1} a_k x^k\right)$$
        \item Det gäller att\sidenote[][]{Denna räkneregel kan förstås generealiseras till att högre potenser av $k$ motsvarar högre derivator -- och om vi istället delar med någon potens av $k$ får vi primitiva funktioner till den genererande funktionen.}
        $$\sum_{k=0}^{\infty} k a_k x^k = \frac{F_a'(x)}{x}.$$
        
    \end{enumerate}
    \begin{proof}
        De första två är tydliga -- mellersta uttrycket är närmast ett bevis av påståendet -- så vi ger enbart ett bevis av det tredje.

        Först observerar vi att $ka_k=0$ när $k=0$, så
        $$\sum_{k=0}^{\infty} k a_k x^k = \sum_{k=1}^{\infty} k a_k x^k$$
        och sedan kommer vi ihåg att $\frac{\intd}{\dx} x^{k} = k x^{k-1}$, vilket ger oss att
        \begin{align*}
            \sum_{k=1}^{\infty} k a_k x^k &= \frac{1}{x}\sum_{k=0}^{\infty} k a_k x^{k-1}\\
            &= \frac{1}{x}\sum_{k=1}^{\infty} a_k \frac{\intd}{\dx} x^k
        \end{align*}
        och vi minns att derivatan är en linjär operator, så vi kan flytta ut den ur summan och få
        \begin{align*}
            \frac{1}{x}\sum_{k=1}^{\infty} a_k \frac{\intd}{\dx} x^k &= \frac{1}{x} \frac{\intd}{\dx} \left(\sum_{k=1}^{\infty} a_k x^k\right)\\
            &= \frac{1}{x}\frac{\intd}{\dx}\left(F_a(x) - a_0\right) = \frac{F_a'(x)}{x}
        \end{align*}
        vilket är vad vi ville bevisa.
    \end{proof}
\end{lemma}

\section{Produkter av genererande funktioner}

Nästa sak vi skall diskutera är vad som händer om vi multiplicerar två genererande funktioner -- alltså vad det motsvarar på den kombinatoriska sidan.

\begin{definition}
    Låt $\{a_k\}_{k=0}^\infty$ och $\{b_k\}_{k=0}^\infty$ vara två följder. \emph{Faltningen} av $a$ och $b$ betecknar vi med $a*b$, och $\{(a * b)_k\}_{k=0}^\infty$ ges av
    $$(a*b)_k = \sum_{i=0}^{k} a_i b_{k-i}.$$
\end{definition}

\begin{example}
    Låt $\{a_k\}_{k=0}^\infty$ vara någon följd, och låt $\{b_k\}_{k=0}^\infty$ vara följden av bara ettor, alltså $b_k = 1$ för alla $k$. Vad blir $a*b$?

    Enligt definitionen är
    $$(a*b)_k = \sum_{i=0}^{k} a_i b_{k-i} = \sum_{i=0}^{k} a_i$$
    så följden $a*b$ är helt enkelt $(a_0, a_0 + a_1, a_0 + a_1 + a_2, \ldots)$, alltså de kumulativa summorna av följden.
\end{example}

\begin{example}
    Låt $\{a_k\}_{k=0}^\infty$ vara någon följd, och låt $\{b_k\}_{k=0}^\infty$ vara följden som börjar med $n$ stycken ettor, och sedan är noll. Alltså $b_k = 1$ om $k \leq n$, $0$ annars. Vad blir $a*b$?

    Enligt definitionen är
    $$(a*b)_k = \sum_{i=0}^{k} a_i b_{k-i} = \sum_{i=k-n}^{k} a_i$$
    så vad vi gör för att räkna ut $(a*b)_k$ är att vi tar summan av de senaste $n$ termerna i $a_k$.
\end{example}

Låt oss nu berätta vad som händer på den algebraiska sidan när vi faltar på den kombinatoriska sidan.

\begin{theorem}\label{convolution_is_product_of_gen_funcs}
    Låt $\{a_k\}_{k=0}^\infty$ och $\{b_k\}_{k=0}^\infty$ vara två följder med generande funktioner $F_a$ och $F_b$. Den genererande funktionen för $a*b$ är då $F_a(x)F_b(x)$. Vi har alltså att
    $$F_{a*b}(x) = F_a(x)F_b(x)$$
    så generande funktioner omvandlar faltningar till produkter.

    \begin{proof}
        Vi bevisar detta algebraiskt, genom att helt enkelt skriva upp vad $F_a(x)F_b(x)$ är för något och manipulera uttrycket tills det blir $F_{a*b}(x)$. Så
        \begin{align*}
            F_a(x)F_b(x) &= \left(\sum_{k=0}^{\infty} a_k x^k\right)\left(\sum_{l=0}^{\infty} b_l x^l\right)\\
            &= \sum_{k, l = 0}^{\infty} a_k b_l x^{k+l}
        \end{align*}
        om vi multiplicerar ut de två stora summorna.\sidenote[][]{Vore det här en kurs i analys hade vi behövt fundera långt och väl på om det faktiskt är tillåtet att multiplicera ut en produkt av oändliga summor på detta viset, och om koefficienterna vi fick verkligen är dessa. Men vi bryr oss inte om vad analytikerna tycker, och bryr oss inte om konvergens -- för oss är dessa formella objekt, så de multiplicerar ut enligt de algebraiska regler vi förväntar oss. Vad detta gör med konvergensen är en fråga för någon annan.}

        Som nästa steg grupperar vi termerna efter exponenten på $x$, och skriver att
        \begin{align*}
            \sum_{k, l = 0}^{\infty} a_k b_l x^{k+l} &= \sum_{j=0}^{\infty} \sum_{\substack{k, l \geq 0\\k + l = j}} a_k b_l x^j\\
            &= \sum_{j=0}^{\infty} \left(\sum_{\substack{k, l \geq 0\\k + l = j}} a_k b_l\right) x^j.
        \end{align*}

        Det här ser ju väldigt mycket ut som en genererande funktion -- specifikt en generande funktion för följden $\{c_j\}_{j=0}^\infty$ där
        $$c_j = \sum_{\substack{k, l \geq 0\\k + l = j}} a_k b_l.$$

        Men denna kan vi ju skriva på ett lite annat sätt -- nämligen
        $$\sum_{\substack{k, l \geq 0\\k + l = j}} a_k b_l = \sum_{i=0}^{j}a_{i} b_{j-i}$$
        och vi ser att detta är faltningen av $a$ och $b$, så vi har bevisat resultatet.
    \end{proof}
\end{theorem}

Låt oss formulera detta resultat också för fler än två följder -- beviset är det samma men med mer notation, så vi utelämnar det.

\begin{lemma}\label{lemma_convolution_many_seqs}
    Antag att $\{a^1_k\}_{k=0}^\infty, \{a^2_k\}_{k=0}^\infty, \ldots, \{a^d_k\}_{k=0}^\infty$ är en samling av $d$ stycken följder. Då gäller det att\sidenote[][]{Den uppmärksamma av er, som läst en kurs i algebra, kanske anmärker här på att vi bara definierat $a * b$, så ett uttryck som $a^1 * a^2 * \ldots * a^d$ bara är väldefinierat om faltning är en associativ operation.
    
    Hav förtröstan, frukta icke, faltningen är inte bara associativ utan också kommutativ. Den till och med distribuerar över addition.}
    $$(a^1*a^2*\ldots*a^d)_k = \sum_{\substack{k_1, k_2, \ldots, k_d \geq 0\\k_1 + k_2 + \ldots + k_d = k}} a^1_{k_1}a^2_{k_2}\ldots a^d_{k_d}$$
    och
    $$F_{a^1 * a^2 * \ldots * a^d}(x) = \prod_{i=1}^{d} F_{a^i}(x).$$
\end{lemma}

Vi tar nu och tillämpar detta maskineri vi byggt upp på ett faktiskt kombinatoriskt problem.

\begin{example}
    Låt $a_k$ vara antalet lösningar till
    $$x_1 + x_2 + x_3 + x_4 + x_5 = k$$
    om vi kräver att alla $x_i$ är ickenegativa heltal. Finn den genererande funktionen till denna följd.

    Låt, för $i = 1,2,\ldots,5$, $a^i$ vara följden av bara ettor, så $a_k^i = 1$ för alla $i$ och $k$.
    
    Vi betraktar nu faltningen $a^1*a^2*a^3*a^4*a^5$. Enligt Lemma \ref{lemma_convolution_many_seqs} är
    $$(a^1*a^2*a^3*a^4*a^5)_k = \sum_{\substack{k_1, k_2, k_3, k_4, k_5 \geq 0\\k_1 + k_2 + k_3 + k_4 + k_5 = k}} a^1_{k_1}a^2_{k_2} a^3_{k_3} a^4_{k_4} a^5_{k_5},$$
    men summanden här är ju så klart alltid bara en produkt av fem ettor, alltså ett. Så
    \begin{align*}
        \sum_{\substack{k_1, k_2, k_3, k_4, k_5 \geq 0\\k_1 + k_2 + k_3 + k_4 + k_5 = k}} a^1_{k_1}a^2_{k_2} a^3_{k_3} a^4_{k_4} a^5_{k_5} &= \sum_{\substack{k_1, k_2, k_3, k_4, k_5 \geq 0\\k_1 + k_2 + k_3 + k_4 + k_5 = k}} 1\\
        &= \abs{\left\{k_1, \ldots, k_5 \geq 0 \given k_1 + \ldots + k_5 = k \right\}}
    \end{align*}
    och denna faltningen är alltså precis lika med $a$, följden vi var ute efter.

    Om vi nu tar likheten $a = a^1*a^2*\ldots*a^5$ och tar genererande funktionen av bägge sidorna får vi likheten $F_a(x) = F_{a^1*a^2*\ldots*a^5}(x)$, och tillämpar vi nu åter Lemma \ref{lemma_convolution_many_seqs} får vi att
    $$F_a(x) = F_{a^1}(x)F_{a^2}(x)\ldots F_{a^5}(x) = \Big(F_{a^1}(x)\Big)^5$$
    där vi i sista steget använde att $a^1 = a^2 = \ldots = a^5$, så de alla har samma genererande funktion.

    Vi såg i början av föreläsningen att den genererande funktionen för en följd av enbart ettor är $\frac{1}{1-x}$, så vad vi fått ut är att
    $$F_a(x) = \left(\frac{1}{1-x}\right)^5.$$
\end{example}

Vi hade så klart gärna haft en explicit formel för antalet lösningar på den här ekvationen. Vi fuskar genom att redan veta vad svaret borde bli, och bevisa att den följden har samma genererande funktion som följden vi just studerade -- detta bevisar alltså att följderna är lika med varandra.

\begin{lemma}
    Fixera något heltal $n\geq 1$, och låt följden $\{a_k\}_{k=0}^\infty$ ges av
    $$a_k = \binom{n+k-1}{k}.$$

    Den genererande funktionen för denna följden är\sidenote[][]{Så när vi visat detta exemplet kommer vi alltså att veta att antalet lösningar till
    $$x_1 + x_2 + x_3 + x_4 + x_5 = k$$
    i ickenegativa heltal är precis
    $$\binom{k+4}{k}$$
    eftersom vi vet att den genererande funktionen för antalet lösningar till den ekvationen var $\frac{1}{(1-x)^5}$.
    
    I just detta fallet visste vi ju dock redan det tack vare vår formel för antalet multi-delmängder.}
    $$F_a(x) = \frac{1}{(1-x)^n}.$$

    \begin{proof}
        Vi bevisar detta med induktion i $n$. För basfallet $n=1$ blir $\binom{n+k-1}{k} = \binom{k}{k} = 1$, så följden är konstant ett, och vi vet sedan innan att genererande funktionen för följden av bara ettor är $\frac{1}{1 - x}$, så basfallet håller.

        För induktionssteget, antag att $n \geq 1$, och vi vill visa att likheten gäller för $n+1$. Vår induktionshypotes är nu att
        $$F_a(x) = \sum_{k=0}^{n} \binom{n+k-1}{k}x^k = \frac{1}{(1 - x)^n}.$$
        
        Alltså har vi att
        $$\frac{\intd}{\dx} F_a(x) = \frac{n}{(1-x)^{n+1}},$$
        så att
        \begin{align*}
            \frac{1}{(1-x)^{n+1}} &= \frac{1}{n}\frac{\intd}{\dx} F_a(x)\\
            &= \frac{1}{n} \frac{\intd}{\dx}\left(\sum_{k=0}^{n} \binom{n+k-1}{k}x^k\right)\\
            &= \frac{1}{n}\sum_{k=0}^{n} \binom{n+k-1}{k}\frac{\intd}{\dx}x^k\\
            &= \frac{1}{n}\sum_{k=1}^{n} \binom{n+k-1}{k}k x^{k-1}.
        \end{align*}
        Notera att vi i sista steget ser att derivatan av $x^0$ är noll, så termen för $k=0$ försvinner. Justerar vi summeringsindex lite grann så ser vi att
        \begin{align*}
            \frac{1}{n}\sum_{k=1}^{n} \binom{n+k-1}{k}k x^{k-1} &= \frac{1}{n}\sum_{k=0}^{n} \binom{n+k}{k+1}(k+1) x^{k}.
        \end{align*}

        Så låt oss studera summanden i det här uttrycket. Vi ser att
        \begin{align*}
            \binom{n+k}{k+1}\frac{k+1}{n} &= \frac{(n+k)!}{(k+1)!(n-1)!}\frac{k+1}{n}\\
            &= \frac{(n+k)!}{k!n!}\\
            &= \binom{n+k}{k} = \binom{(n+1) + k - 1}{k}
        \end{align*}
        så om vi sätter tillbaka detta i vår summa ser vi att vad vi fått är att
        $$\frac{1}{(1-x)^{n+1}} = \sum_{k=0}^{\infty} \binom{(n+1) + k - 1}{k} x^k$$
        vilket är precis vad vi ville bevisa.
    \end{proof}
\end{lemma}

\section{Övningar}

\begin{xca}
    Antag att en följd $\{a_k\}_{k=0}^\infty$ lyder en rekurrensrelation
    $$a_{k+1} = \sum_{i = 0}^{\ell} w_i a_{k-i}$$
    för alla $k \geq \ell$, där $(w_0, w_1, w_2, \ldots, w_\ell)$ är en följd med koefficienter.\sidenote[][]{Om det hjälper kan ni anta att $w_i = 0$ för $i \not \in 0,1,\ldots,\ell$.}

    Använd våra räkneregler i Lemma \ref{lemma_generating_function_calc_rules} för att finna ett uttryck för den genererande funktionen $F_a$.\sidenote[][]{Det här är alltså generaliseringen av vad vi gjorde för Fibonnaciföljden. Ni kommer finna att ni kan skriva svaret som en rationell funktion med koefficienter som ges av de första termerna i $a_k$ och av vikterna $w_i$.}
\end{xca}

\begin{xca}
    Antag att vi har en summa
    \begin{equation}\label{exercise_annoying_double_sum}
        \sum_{k=\ell}^{\infty} \sum_{i=k-\ell}^{k} f(k,i)
    \end{equation}
    för någon funktion $f: \N^2\to \R$. Byt ordningen på summeringen,\sidenote[][]{Det här är väl egentligen bara en övning i att manipulera summor, inte i kombinatorik per se. Men det är användbart att kunna trick som det här för att arbeta med genererande funktioner. Ibland kan ``byt ordningen i en summa'' till och med vara en bevisteknik i sig.} alltså fyll i frågetecknen i uttrycket
    $$\sum_{i = 0}^{\infty} \sum_{k = ?}^{?} f(?,?)$$
    så att det blir lika med vad vi hade i \eqref{exercise_annoying_double_sum}.\sidenote[][]{Tips: För varje par av $(k,i)$, hur många gånger dyker $f(k,i)$ upp i summan i \eqref{exercise_annoying_double_sum}? Det hjälper att rita upp en tabell över par av $i$ och $k$, och fylla i varje ruta om det paret dyker upp. Om vi har värden på $k$ som rader och värden på $i$ som kolumner räknar \eqref{exercise_annoying_double_sum} ``radvis'', och när vi byter ordning på summan vill vi i stället räkna ``kolumnvis''.}
\end{xca}

\begin{xca}
    Låt, om $\{a_k\}_{k=0}^\infty$ och $\{b_k\}_{k=0}^\infty$ är två följder, $a+b$ vara följden som ges av $(a+b)_k = a_k + b_k$. Bevisa att $F_{a+b}(x) = F_a(x) + F_b(x)$, så att addition på den kombinatoriska sidan motsvarar addition också på den algebraiska sidan.
    
    Bevisa sedan, om $\{c_k\}_{k=0}^\infty$ är en tredje följd, att
    $$a*(b + c) = a*b + a*c,$$
    det vill säga att faltning distribuerar över addition.\sidenote[][]{Ledtråd: Använd genererande funktioner, specifikt vad ni bevisade i första halvan av uppgiften och Teorem \ref{convolution_is_product_of_gen_funcs}.}
\end{xca}

\begin{xca}
    Låt $f_k$ vara följden av Fibonaccital, såsom vi definierat dem. Låt följden $g_k$ vara $(1, -1, -1, 0, 0, 0,\ldots)$.

    \begin{enumerate}
        \item Vad är den genererande funktionen för $\{g_k\}_{k=0}^\infty$?
        \item Vad är faltningen $f * g$? Räkna ut detta som om ni inte visste vad den genererande funktionen till $f$ är, alltså utan att använda er av Teorem \ref{convolution_is_product_of_gen_funcs}.
        \item Använd vad ni gjorde i de första två delfrågorna och Teorem \ref{convolution_is_product_of_gen_funcs} för att ge ett alternativt bevis för att
        $$F_f(x) = \frac{1}{1 - x - x^2}.$$
    \end{enumerate}
\end{xca}

\begin{xca}
    Använd genererande funktioner\sidenote[][-1cm]{Ledtråd: Betrakta den här summan som en faltning av två följder -- en av dem kan vi redan den genererande funktionen för, den andra är enkel att finna den genererande funktionen för givet vad vi redan kan. Använd sedan Teorem \ref{convolution_is_product_of_gen_funcs} för att få ut likheten.} för att bevisa att
    $$\sum_{i=0}^{k} (-1)^i \binom{n}{i}\binom{n + k - i - 1}{k - i} = 0$$
    för alla $k \geq 1$.\sidenote[][]{Frivillig bonus om ni har tråkigt någon dag: Kan ni komma på ett kombinatoriskt bevis för den här likheten? Jag funderade på det en kort stund och fann inget.}
\end{xca}

\begin{xca}
    Finn enkla uttryck för de genererande funktionerna för följderna
    $$a_k = 3^k, \qquad b_k = \left(\frac{1}{4}\right)^k$$
    och
    $$c_k = \frac{5}{k!}.$$
\end{xca}

%\bibliography{references}
%\bibliographystyle{plainnat}

\end{document}
