\documentclass{tufte-handout}

\title{Kombinatorik 1MA020: Information om kursen}

\author[Vilhelm Agdur]{Vilhelm Agdur\thanks{\href{mailto:vilhelm.agdur@math.uu.se}{\nolinkurl{vilhelm.agdur@math.uu.se}}}}

\date{15 januari 2024}

%\geometry{showframe} % display margins for debugging page layout

\usepackage{graphicx} % allow embedded images
  \setkeys{Gin}{width=\linewidth,totalheight=\textheight,keepaspectratio}
  \graphicspath{{graphics/}} % set of paths to search for images
\usepackage{amsmath}  % extended mathematics
\usepackage{booktabs} % book-quality tables
\usepackage{units}    % non-stacked fractions and better unit spacing
\usepackage{multicol} % multiple column layout facilities
\usepackage{lipsum}   % filler text
\usepackage{fancyvrb} % extended verbatim environments
  \fvset{fontsize=\normalsize}% default font size for fancy-verbatim environments

\usepackage{color,soul} % Highlights for text

\include{mathcommands.extratex}

\begin{document}

\maketitle% this prints the handout title, author, and date

\begin{abstract}
\noindent
Den här filen innehåller -- förhoppningsvis -- all praktisk information om kursen som du kan tänkas behöva.\sidenote{Om du saknar någon information här, säg till så lägger jag till den.} Hur bonuspoängen fungerar, innehåll
i föreläsningarna och rekommenderade uppgifter, et cetera. 
\end{abstract}

Den listade kurslitteraturen för denna kurs är \emph{Applied Combinatorics}\cite{mainTextbook}. Vi förhåller oss dock väldigt löst till boken -- vi kommer säga många saker som den boken inte säger, och också hoppa över mycket som finns i boken. Den kan vara ett ställe att se en annan presentation av samma material när boken och kursen överlappar, men förlita er inte huvudsakligen på den.

Det huvudsakliga materialet för kursen är föreläsningsanteckningarna, som finns på kurshemsidan. Dessa kommer att putsas till lite grann under kursens gång, men innehållet kommer vara väsentligen det samma.\sidenote[][]{Med undantag för den sista delen av kursen, som täcker sannolikhetsteori och den probabilistiska metoden -- denna del kommer att få större förändringar och förbättringar.}

\section{Planering av föreläsningarna}

Vi skall totalt ha femton schemalagda tillfällen, varav tolv kommer vara föreläsningar, och tre kommer vara övningstillfällen. Om inget oförutsett inträffar så stämmer tiderna och datumen angedda i tabellen i detta dokumentet, men förlita er hellre på TimeEdit.

Innehållet i de sista tre föreläsningarna är fortfarande aningen flytande, och lär ordnas om en del.

\begin{table}[h]
\begin{tabularx}{\textwidth}{llX}
Datum & Tid      & Planerat innehåll \\ 
\midrule
Tis 16 Jan &10:15 & F1: Permutationer och kombinationer, och kombinatoriska bevis\\
Fre 19 Jan &08:15 & F2: Binomialsatsen, kompositioner, multinomialsatsen, och lådprincipen\\
Mån 22 Jan &10:15 & F3: Inklusion-exklusion, derangemang, surjektioner, och mängdpartitioner\\
Tors 25 Jan &08:15 & Övning\\
Tis 30 Jan &10:15 & F4: Sammanfattning av alla räkneproblem, samt cykler\\
Tors 1 Feb &08:15 & F5: Genererande funktioner\\
Mån 5 Feb &13:15 & F6: Fortsättning på genererande funktioner\\
Tors 8 Feb &13:15 & F7: Dyck-stigar och Catalantal\\
Mån 12 Feb &10:15 & Övning\\
Tors 15 Feb &08:15 & F8: Grafer och träd\\
Ons 21 Feb &10:15 & F9: Diskret sannolikhetsteori, introduktion\\
Tors 22 Feb &10:15 & Övning\\
Tis 27 Feb &10:15 & F10: Slumpvariabler\\
Ons 28 Feb &15:15 & F11: Probabilistiska metoden (Också näst sista dag att registrera sig för tentan)\\
Mån 4 Mars &15:15 & Övning
\end{tabularx}
\end{table}

\section{Tentan}

Ordinarie tentamen för kursen inträffar den 11 mars -- kom ihåg att registrera er för den minst tolv dagar i förväg, alltså torsdagen den 29 februari,\sidenote[][]{Det är skottår i år!} det är inget kul att inte få skriva tentan för att man glömt registrera sig.\sidenote{Det hände mig mer än en gång under min kandidat och master...} Det går att få max fyrtio poäng på tentan, och för att få betyget 5a, 4a, 3a behövs respektive 32/25/18 poäng.

Omtentor i kursen går den 7e juni och i augusti.

\section{Bonuspoäng till tentan}

Det finns tre olika sätt att få bonuspoäng till tentan, och du kan maximalt få sju bonuspoäng.\sidenote[][]{Ni kan alltså kombinera olika sätt att få poäng för att nå upp till totalt högst sju -- om ni gör saker som summerar till mer än sju bonuspoäng får ni ändå bara de sju.}

\begin{enumerate}
	\item Att rätta substantiella fel i föreläsningsanteckningarna, lägga till fler figurer till dem, eller i övrigt förbättra dem.
	\item Att lösa övningarna eller göra programmeringsuppgiften från ett av övningstillfällena.
	\item Att skriva en sammanfattning av en forskningsartikel inom kombinatorik.
\end{enumerate}

Den första av dessa, rättningar av anteckningarna, görs individuellt, och du kan få upp till tre poäng för det -- ett per gång du gör en sådan korrigering eller förbättring. Om du lägger till en till figur skall den självklart faktiskt bidra till att göra materialet lättare att förstå -- en bild bara för bildens skull. Med att ``i övrigt förbättra dem'' menar jag att förtydliga resonmenangen eller ge bättre förklaringar av materialiet. Om du hittar en annan källa än föreläsningsanteckningarna som förklarar samma koncept på ett sätt som du tycker är tydligare kan du alltså få bonuspoäng för att skriva in denna förklaring i föreläsningsanteckningarna.\sidenote[][-0.5cm]{Självklart kan bara en person få poäng per sak som fixas -- om en person hittar ett fel och berättar för sina fem kompisar om det felet, och alla lämnar in en fix av det, så får bara den första som lämnar in det poäng för det.}

Till varje av våra tre övningstillfällen kommer vi att ha en samling uppgifter att lösa, samt ett förslag på ett smärre programmeringsprojekt relaterat till ämnet.\sidenote[][]{Eftersom ämnet för denna kurs är kombinatorik, inte programmering, är det okej att lösa denna genom att be t.ex. ChatGPT att skriva de olika funktionerna ni behöver åt er. Detta är så klart inte lika bra för att faktiskt lära sig programmering, och ni måste skriva att ni gjort det i er rapport om ni gör så.} Ni kan få upp till fyra bonuspoäng genom att skriva lösningar på uppgifterna, och/eller upp till sex bonuspoäng på att lösa programmeringsuppgiften -- programmeringen kräver både kod och en förklarande rapport om era resultat.

Slutligen kan ni få upp till sju bonuspoäng till tentan genom att välja en forskningsartikel inom kombinatorik eller en tillämpning av kombinatorik, läsa den, och skriva en kort\sidenote[][]{Förslagsvis runt fyra-fem sidor.} rapport om vad det är för problem som studeras i artikeln, vad de får för resultat, och en skiss av vilken metod de använder. Jag kommer att lägga upp några olika länkar till var man kan hitta sådana artiklar, men jag kommer också gärna med förslag på andra artiklar om ni har något särskilt ämne ni är intresserade av eller bara tycker det är svårt att välja.

De två uppgifterna som görs i grupp skall göras i grupper av inte mer än fem personer, och helst inte färre än fyra.\sidenote[][]{Om antalet studenter som vill göra uppgiften inte är en jämn multipel av fem får vi så klart göra undantag -- men om det är två grupper av två studenter så \emph{kommer} jag att slå ihop dem.} Dessa skall sedan presenteras av gruppen muntligen på ett övningstillfälle, då ni också får feedback på ert arbete.\sidenote[][]{Att deltaga i denna presentation är \emph{obligatoriskt} för att få godkänt individuellt -- om du inte kan delta när din grupp presenterar måste du informera mig om detta via mail innan den gör det, och vi arrangerar då en annan tid för dig att göra det.} Ni behöver inte ha en powerpoint eller dylikt, men ni ska kunna berätta vad ni har gjort och svara på frågor om det.

\section{Deadlines}

\begin{enumerate}
	\item Förbättringar av föreläsningsanteckningarna är möjliga ända fram till onsdagen den sjätte mars.
	\item För uppgifterna eller programmeringsuppgiften är deadline två veckor efter övningstillfället faktiskt hölls, förutom om man väljer det sista övningstillfället, i vilket fall deadline är fredagen den första mars.\sidenote[][]{Detta är för att övningstillfället faller så sent i kursen -- er uppgift måste presenteras muntligen redan den fjärde mars, och jag behöver tid att faktiskt läsa igenom ert arbete innan ni presenterar det.}
	\item För forskningsartikel-rapporten är deadline fredagen den första mars.
\end{enumerate}

Förlängningar av deadline är möjliga för gruppuppgiften om ni har ett gott skäl att be om det, och ber om det via mail \emph{innan deadline}, så länge ni inte vill ha tid som löper till efter tentaperioden.

\section{Indelning i grupper}

Ni får själva bilda grupper för att göra uppgiften, och sedan meddela mig om gruppens medlemmar och vilken uppgift ni avser göra, så skapar jag en sådan grupp i Studium. Att flytta sig mellan grupper är möjligt fram till två veckor innan deadline för uppgiften -- maila mig om ni av någon anledning vill byta grupp.

\section{Format för inlämningar}

\begin{enumerate}
	\item Förändringar av föreläsningsanteckningarna görs direkt i \LaTeX-källan för anteckningarna, genom en pull request mot det GitHub-repository som de ligger i.\sidenote[][]{Det finns här: \url{https://github.com/vagdur/Kombinatorik-1MA020}
	
	Om du behöver någon hjälp med att lista ut det tekniska med att få detta att fungera får du gärna be om det under ett övningstillfälle eller i pausen på en föreläsning.}
	\item Även uppgiftslösningar och forskningsartikel-rapporten skall vara skrivna i \LaTeX, och rapport-biten av programmeringen likaså. Dessa kan inlämnas antingen som en pull request på GitHub, där ni lägger ert arbete i nya filer,\sidenote[][]{Gärna i en ny mapp om det är mer än en fil ni lämnar in.} eller på Studium.
	\item Programmeringen får göras i valfritt programspråk, så länge det är möjligt för mig att köra er kod på min laptop och verifiera att den gör vad ni påstår.\sidenote[][]{I praktiken betyder väl detta mest bara att ni inte kan använda Mathematica eller Matlab, eftersom de är proprietära system.
	
	Om ni är fundersamma på om ert projekt har blivit för konstigt för min dator att kunna köra det, fråga i pausen på en föreläsning.}
\end{enumerate}

\bibliography{references}
\bibliographystyle{plainnat}



\end{document}
