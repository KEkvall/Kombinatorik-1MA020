\documentclass[nobib]{tufte-handout}

\title{Övningstillfälle 1: Programmeringsövningar $\cdot$ 1MA020}

\author[Vilhelm Agdur]{Vilhelm Agdur\thanks{\href{mailto:vilhelm.agdur@math.uu.se}{\nolinkurl{vilhelm.agdur@math.uu.se}}}}

\date{22 februari 2023}


%\geometry{showframe} % display margins for debugging page layout

\usepackage{graphicx} % allow embedded images
  \setkeys{Gin}{width=\linewidth,totalheight=\textheight,keepaspectratio}
  \graphicspath{{graphics/}} % set of paths to search for images
\usepackage{amsmath}  % extended mathematics
\usepackage{booktabs} % book-quality tables
\usepackage{units}    % non-stacked fractions and better unit spacing
\usepackage{multicol} % multiple column layout facilities
\usepackage{lipsum}   % filler text
\usepackage{fancyvrb} % extended verbatim environments
  \fvset{fontsize=\normalsize}% default font size for fancy-verbatim environments

\usepackage{color,soul} % Highlights for text


\include{mathcommands.extratex}

\begin{document}

\maketitle% this prints the handout title, author, and date

\begin{abstract}
\noindent
Detta dokument innehåller en samling övningar i kombinatorik som kräver programmering: Vi utforskar alltså några olika saker med hjälp av stora mängder beräkningar.
\end{abstract}

\section{Träd och Prüferkoder}

\begin{xca}
    Skriv en funktion som, givet ett träd,\sidenote[][]{Använd förslagsvis ett färdigt paket för grafer för att representera ditt träd, så att du får funktioner för att plotta din graf gratis.} räknar ut dess Prüferkod.
\end{xca}

\begin{xca}
    Skriv en funktion som, givet en Prüferkod, genererar dess korresponderande träd.
\end{xca}

Med dessa funktioner skrivna har vi nu ett naturligt sätt att generera ett slumpmässigt träd på $n$ noder, genom att generera en slumpmässig Prüferkod.

\begin{xca}
    Skriv en funktion som genererar ett slumpmässigt träd på $n$ noder med denna metod, och bevisa att denna kommer generera varje träd med samma sannolikhet.\sidenote[][]{Här har vi alltså lite bevis i vår programmering -- vilket väl är naturligt, om vi vill vara säkra på att vår kod fungerar.}
\end{xca}

Det här är inte den enda naturliga metoden för att generera ett slumpmässigt träd. Låt oss också studera en annan modell för ett slumpmässigt träd.

\begin{xca}
    Antag att $X$ är en slumpvariabel som tar värdet $i$ med sannolikhet $p_i$, för varje $i \in \Z^{\geq 0}$.

    Vi skapar oss ett slumpmässigt träd genom att tänka oss att vi startar med en individ, som sedan får ett antal barn som dras från fördelningen till $X$. Sedan får varje av dessa barn också ett antal barn givet av denna fördelning, oberoende av varandra och sina förfäder.

    Skriv en funktion som, givet en annan funktion som genererar ett slumptal,\sidenote[][]{Om ni skriver i ett språk som varken har funktioner som första klassens objekt eller pekare till funktioner får ni nog ta vektorn $(p_0, p_1, \ldots)$ som argument i stället.} genererar ett slumpmässigt träd på detta vis. Ett träd av detta slag kallas för ett \emph{Galton-Watson-träd}.
\end{xca}

\begin{xca}
    Givet ett rotat träd $T$ kan vi kan definiera en slumpvariabel $X$ genom att slumpmässigt välja en nod i trädet, och låta $X$ vara antalet barn till denna nod.

    Generera ett par olika slumpmässiga träd likformigt, med metoden med Prüferkoder, och generera sedan ett Galton-Watson-träd med hjälp av det $X$ som motsvarar dessa träd. Rita upp träden sida vid sida. Ser de ut som ungefär samma träd eller är de radikalt annorlunda?
\end{xca}

\begin{xca}
    Låt oss definiera ett par olika parametrar för ett träd:
    \begin{enumerate}
        \item Dess \emph{gradföljd} $(d_1,d_2,\ldots,d_n)$ ges av att $d_i$ är antalet grannar till nod $i$.
        \item Dess \emph{genomsnittliga grad} $\bar{d}$ är genomsnittet av antalet grannar hos en nod i trädet, det vill säga $\bar{d} = \frac{1}{n}\sum_{i=1}^{n} d_i$.
        \item Dess \emph{maximala} och \emph{minimala} grad är maximum respektive minimum av $d_i$.
        \item Dess \emph{diameter} är det största avståndet mellan två noder i trädet, där vi med avståndet mellan $u$ och $v$ menar att om vi har en följd av distinkta noder $u = v_0 \, v_1\, v_2\ldots v_{k-1}\, v_k = v$ sådan att $v_{i-1}$ och $v_i$ är grannar för varje $i$ så är avståndet mellan dem $k$.\sidenote[][]{I ett träd finns det ju alltid exakt en sådan följd, så att denna definition är otvetydig. I en allmän graf menar vi längden på den \emph{kortaste} sådana följden.}
    \end{enumerate}

    Låt nu $X$ vara en Poissonfördelad slumpvariabel med parameter $\frac{3}{4}$,\sidenote[][]{Det vill säga,$$\Prob{X = k} = \frac{(3/4)^k e^{-3/4}}{k!}.$$} och låt $T_{GW}$ vara ett Galton-Watson-träd med $X$ som fördelning för antalet barn. Låt $T_{Unif}$ vara ett likformigt slumpmässigt valt träd på lika många noder som $T_{GW}$.

    Simulera dessa par av träd ett stort antal gånger, och för varje av de parametrar vi definierade ovan, rita upp ett histogram över dess värde för de två olika träden. Inkludera medelvärdet för parametern för de två olika träden i rubriken till histogrammet.
\end{xca}

\section{Oberoende mängder i grafer och den giriga algoritmen}

I vårt bevis av Caro-Weis sats om oberoende mängder i grafer finns det implicit en algoritm för att hitta en sådan: Välj en slumpmässig etikettering av grafen, och låt sedan vår oberoende mängd vara mängden av noder som har en lägre etikett än alla sina grannar.

Det finns också en relaterad, ännu enklare algoritm, den så kallade giriga algoritmen: Givet en etikettering av din graf, gå igenom dess noder i ordning från minst till störst. Varje gång du ser en nod som inte har några grannar redan i din oberoende mängd, lägg till noden du nu är på till din oberoende mängd.

\begin{xca}
    Bevisa att du, ifall du tar din slumpmässiga ordning från Caro-Wei-algoritmen och i stället kör den giriga algoritmen, alltid kommer att få en oberoende mängd som innehåller den som Caro-Wei gav dig, men som kan vara större.
\end{xca}

\begin{xca}
    Implementera dessa bägge algoritmer.
\end{xca}

\begin{xca}
    Vi vet från föreläsningarna att en Erd\H{o}s-Rényi-graf $G(n,p)$ har $n$ noder, och varje möjlig kant är med i grafen oberoende med sannolikhet $p$. Skriv en funktion som givet $n$ och $p$ genererar en sådan slumpmässig graf.
\end{xca}

\begin{xca}
    Generera ett stort antal $G\left(10000, \frac{15}{10000}\right)$-grafer. För varje av dem, kör Caro-Wei-algoritmen, den giriga algoritmen med den slumpmässiga ordning ni använde för Caro-Wei, och den giriga algoritmen i ordning från lägst grad till högst. Kalla storlekarna på de oberoende mängderna dessa hittar för $\alpha_{CW}(G)$, $\alpha_{GU}(G)$, och $\alpha_{GO}(G)$.

    Rita upp ett histogram över varje av dessa tre tal och räkna ut deras medelvärden. Räkna också ut medelvärdet av $\frac{\alpha_{GU}(G)}{\alpha_{CW}(G)}$ -- notera att detta \emph{inte} är samma sak som att räkna ut medelvärdena av de två talen och sedan genomföra divisionen.
\end{xca}

\textbf{Bonus:} Inga av dessa algoritmer är i närheten av optimala. Kan ni komma på någon algoritm som presterar bättre? Implementera den och se i praktiken hur mycket större mängderna den finner blir än vad den giriga algoritmen finner.

%\bibliography{references}
%\bibliographystyle{plainnat}

\end{document}
