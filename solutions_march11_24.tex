\documentclass[nobib]{tufte-handout}

\title{Lösningsförslag för tentamen i kombinatorik, 11 Mars 2024 $\cdot$ 1MA020}

\author[Vilhelm Agdur]{Vilhelm Agdur\thanks{\href{mailto:vilhelm.agdur@math.uu.se}{\nolinkurl{vilhelm.agdur@math.uu.se}}}}

\date{19 Mars 2024}


%\geometry{showframe} % display margins for debugging page layout

\usepackage{graphicx} % allow embedded images
  \setkeys{Gin}{width=\linewidth,totalheight=\textheight,keepaspectratio}
  \graphicspath{{graphics/}} % set of paths to search for images
\usepackage{amsmath}  % extended mathematics
\usepackage{booktabs} % book-quality tables
\usepackage{units}    % non-stacked fractions and better unit spacing
\usepackage{multicol} % multiple column layout facilities
\usepackage{lipsum}   % filler text
\usepackage{fancyvrb} % extended verbatim environments
  \fvset{fontsize=\normalsize}% default font size for fancy-verbatim environments

\usepackage{color,soul} % Highlights for text


\include{mathcommands.extratex}
\setlength{\extrarowheight}{12pt}

\begin{document}

\definecolor{darkgreen}{rgb}{0.0627, 0.4588, 0.1451}

\maketitle% this prints the handout title, author, and date

\begin{abstract}
\noindent

Denna fil ger lösningar på uppgifterna i tentan den 15de mars 2023.

\end{abstract}

\section{Fråga 1}

Detta är sats 24 ur föreläsning 2.

\section{Fråga 2a)}

Vi tänker oss att vi har en grupp av $n$ personer, och vi skall välja någon delgrupp av godtycklig storlek, som skall ha en ledare.

I vänster led tänker vi oss att vi börjar med att välja storleken på gruppen, och kallar den $k$. När vi har valt storleken väljer vi specifikt vilka $k$ individer vi skall inkludera, vilket kan göras på ${n \choose k}$ sätt. Sedan väljer vi en ledare ur gruppen, vilket vi kan göra på $k$ sätt.

I höger led tänker vi oss att vi börjar med att välja en ledare för gruppen, vilket kan göras på $n$ sätt, och sedan väljer vi resten av medlemmarna i gruppen. Dessa återstående medlemmar är en delmängd till mängden av återstående personer, vilka är $n-1$ stycken, och vi vet att det finns $2^{n-1}$ delmängder till en mängd med $n-1$ element. Alltså har vi $2^{n-1}$ val av resten av medlemmarna i gruppen.

\section{Fråga 2b)}

Vi har $n$ stycken personer som skall hålla en fest -- de flesta kan bara slappa, men $k$ stycken behöver hjälpa till. Av dessa skall två handla, tre duka, och resten diska efter festen.

I vänster led väljer vi först de två personerna som skall handla, vilket vi kan göra på $n \choose 2$ sätt. Sedan har vi $n-2$ personer kvar att välja mellan för de tre som skall duka, så dem kan vi välja på $n - 2 \choose 3$ sätt. Till sist skall vi välja de $k-5$ personerna som skall diska bland de återstående $n-5$ personerna, vilket vi gör på $n - 5 \choose k - 5$ sätt.

I höger led väljer vi istället först vilka $k$ personer som skall hjälpa till alls, vilket kan göras på $n \choose k$ sätt. Sedan väljer vi vilka två av dessa som skall handla, vilket göres på $k \choose 2$ sätt, och vilka tre som skall duka, vilket vi kan göra på $k - 2 \choose 3$ sätt. Resten får diska.

\section{Fråga 3a)}

\begin{definition}
    Om $\{a_k\}_{k=0}^\infty$ är någon följd ges dess \emph{exponentiella genererande funktion} av
    $$EG_a(x) = \sum_{k=0}^{\infty} a_k \frac{x^k}{k!}.$$
\end{definition}
 
\section{Fråga 3b)}

Fibonaccitalen $\{f_i\}_{i=0}^\infty$ definieras rekursivt av att $f_0 = f_1 = 1$, och
$$f_{k+2} = f_{k+1} + f_k, \quad \forall k \geq 0.$$

Multiplicerar vi denna likhet med $\frac{x^k}{k!}$ på bägge sidor, och sedan summerar detta över alla $k$ för vilken likheten gäller -- det vill säga alla $k \geq 0$ -- så får vi att
\begin{equation}\label{q3a:1}
    \sum_{k=0}^{\infty} f_{k+2}\frac{x^k}{k!} = \sum_{k=0}^{\infty} f_{k+1}\frac{x^k}{k!} + \sum_{k=0}^{\infty} f_{k}\frac{x^k}{k!}.
\end{equation}

Vi observerar nu att
$$\frac{x^k}{k!} = \frac{\intd}{\dx} \frac{x^{k+1}}{(k+1)!} = \frac{\intd^2}{\dx^2} \frac{x^{k+2}}{(k+2)!},$$
så om vi använder detta i \eqref{q3a:1}, och observerar att derivatan är linjär så vi får plocka ut den ur summan, så får vi att
$$\frac{\intd^2}{\dx^2}\left(\sum_{k=0}^{\infty} f_{k+2}\frac{x^{k+2}}{(k+2)!}\right) = \frac{\intd}{\dx}\left(\sum_{k=0}^{\infty} f_{k+1}\frac{x^{k+1}}{(k+1)!}\right) + \sum_{k=0}^{\infty} f_{k}\frac{x^k}{k!}.$$

Uttrycken som står inuti derivatorna nu är ju nästan exakt den exponetiella genererande funktionen -- förutom att den med andraderivatan saknar sin konstanta och linjära term, och den med förstaderivatan saknar sin konstanta term.

Men vi vet ju att derivatan oavsett skickar just dessa termer på noll -- så vi kan lägga till dem utan att ändra uttryckets värde! Alltså är vad vi har sett att
$$EG_f''(x) = EG_f'(x) + EG_f(x),$$
så den differentialekvation som den exponentiella genererande funktionen lyder är alltså $y'' = y' + y$.

Vill vi ha randvillkor för denna differentialekvation kan vi direkt räkna ut att vi måste ha $EG_f(0) = f_0 = 1$ och $EG_f'(0) = f_1 = 1$.

Att faktiskt finna en lösning till denna differentialekvation var inte en del av uppgiften, och att göra det gav inte mer poäng på tentan. För den intresserade kan vi ändock nämna att lösningen är\sidenote[][]{Jämför här Binets formel för Fibonaccitalen, som också innehållet många rötter ur fem, och det gyllene snittet.}
$$y(x) = \frac{1}{10} e^{-\frac{1}{2} (\sqrt{5} - 1) x} \left((5 + \sqrt{5}) e^{\sqrt{5} x} + 5 - \sqrt{5}\right).$$

\section{Fråga 4 a \& b}

Se föreläsning 9 och 10.

\section{Fråga 5 a \& b}

Del a i denna fråga blev lite för vag för att kunna rättas ordentligt, så den ströks ur tentan.

Oavsett, se föreläsning 4 för bägge delarna av denna uppgift.

\section{Fråga 6a)}

En Erd\H{o}s-Rényi-graf med parameterar $n$ och $p$ är en graf på $n$ noder, där varje möjlig kant är med med sannolikhet $p$, oberoende av varandra.

Vi kan skapa en $k$-cykel genom att först välja en mängd av $k$ stycken noder, och sedan välja en cyklisk ordning\sidenote[][]{Med detta menar vi alltså en permutation av dem, men vi betraktra till exempel $1234$ och $4123$ som samma ordning, eftersom de ser likadana ut när vi skriver dem i en cirkel -- tänk på exemplet tidigt i kursen med att placera människor kring ett runt bord.} av dem. Så låt, för varje mängd $A \in {[n] \choose k}$ och varje cyklisk ordning $\pi$, $C_{A,\pi}$ vara händelsen att noderna i $A$ bildar en cykel i ordning $\pi$.

Om vi låter $X$ vara antalet $k$-cykler i vår graf har vi alltså att
$$X = \sum_{A, \pi} \ind{C_{A,\pi}}$$,
så vi kan med väntevärdets linjäritet räkna ut att
$$\E{X} = \E{\sum_{A, \pi} \ind{C_{A,\pi}}} = \sum_{A, \pi} \E{\ind{C_{A,\pi}}} = \sum_{A, \pi} \Prob{C_{A,\pi}}.$$

Vi ser enkelt att sannolikheten för varje given händelse $C_{A,\pi}$ måste vara precis $p^k$, eftersom det är exakt $k$ stycken kanter som måste vara med i grafen för att de noderna skall bilda en cykel i den ordningen.

Vi vet också att det finns $n \choose k$ sätt att välja $A$, och det finns $(k-1)!$ sätt att välja $\pi$.\sidenote[][]{Det finns $k!$ sätt att välja en permutation, men sedan finns det $k$ stycken olika permutationer som motsvarar varje given cyklisk ordning, genom att välja en plats längs det runda bordet att börja skriva permutationen. Se exemplet i början av kursen om att placera personer runt runda bord.} Alltså har vi att
$$\E{X} = \sum_{A, \pi} \Prob{C_{A,\pi}} = {n \choose k}(k-1)!p^k = \frac{n!}{k (n-k)!} p^k.$$

\section{Fråga 6b)}

Markovs olikhet säger oss att $\Prob{X > 0} \leq \E{X}$, så för att visa att sannolikheten att det finns en $k$-cykel går mot noll räcker det att visa att väntevärdet vi räknade ut ovan går mot noll. Om vi stoppar in $p = \frac{c}{n}$ i den räkningen får vi att vi vill visa att
$$\lim_{n \to \infty} \frac{n! c^k}{n^k k (n-k)!}$$

%\bibliography{references}
%\bibliographystyle{plainnat}

\end{document}
