\documentclass[nobib]{tufte-handout}

\title{Lösningsförslag för tentamen i kombinatorik, 15 Mars 2023 $\cdot$ 1MA020}

\author[Vilhelm Agdur]{Vilhelm Agdur\thanks{\href{mailto:vilhelm.agdur@math.uu.se}{\nolinkurl{vilhelm.agdur@math.uu.se}}}}

\date{15 mars 2023}


%\geometry{showframe} % display margins for debugging page layout

\usepackage{graphicx} % allow embedded images
  \setkeys{Gin}{width=\linewidth,totalheight=\textheight,keepaspectratio}
  \graphicspath{{graphics/}} % set of paths to search for images
\usepackage{amsmath}  % extended mathematics
\usepackage{booktabs} % book-quality tables
\usepackage{units}    % non-stacked fractions and better unit spacing
\usepackage{multicol} % multiple column layout facilities
\usepackage{lipsum}   % filler text
\usepackage{fancyvrb} % extended verbatim environments
  \fvset{fontsize=\normalsize}% default font size for fancy-verbatim environments

\usepackage{color,soul} % Highlights for text

% Standardize command font styles and environments
\newcommand{\doccmd}[1]{\texttt{\textbackslash#1}}% command name -- adds backslash automatically
\newcommand{\docopt}[1]{\ensuremath{\langle}\textrm{\textit{#1}}\ensuremath{\rangle}}% optional command argument
\newcommand{\docarg}[1]{\textrm{\textit{#1}}}% (required) command argument
\newcommand{\docenv}[1]{\textsf{#1}}% environment name
\newcommand{\docpkg}[1]{\texttt{#1}}% package name
\newcommand{\doccls}[1]{\texttt{#1}}% document class name
\newcommand{\docclsopt}[1]{\texttt{#1}}% document class option name
\newenvironment{docspec}{\begin{quote}\noindent}{\end{quote}}% command specification environment

\include{mathcommands.extratex}
\setlength{\extrarowheight}{12pt}

\begin{document}

\definecolor{darkgreen}{rgb}{0.0627, 0.4588, 0.1451}

\maketitle% this prints the handout title, author, and date

\begin{abstract}
\noindent

\end{abstract}

\section{Lösningsförslag till tentan}

\section{Fråga 1, del a}

Vi vill se varför denna formel för $f_{n,k}$ räknar etiketterade skogar med $n$ noder på $k$ träd. Vill vi skapa oss en sådan skog så börjar vi med att bestämma hur många noder varje träd skall innehålla -- vi säger att träd ett har $n_1$ noder, träd två har $n_2$ noder, och så vidare.

Sedan fördelar vi ut de $n$ etiketterna i dessa $k$ olika högar, så att varje hög får rätt antal etiketter -- detta kan göras på $\binom{n}{n_1, n_2, \ldots, n_k}$ sätt.

Sedan väljer vi, för varje hög av etiketter, vilket träd med dessa etiketter vi vill ha. Cayleys formel ger oss att detta kan, för varje träd, göras på $n_i^{n_i-2}$ sätt, så multiplikationsprincipen ger oss att vi kan välja vår skog på $n_1^{n_1-2}n_2^{n_2-2}\ldots n_k^{n_k-2}$ sätt.

Hittills är formeln vi har sett
$$\sum_{\substack{n_1, n_2, \ldots, n_k\\ n_i \geq 1\\n_1 + n_2 + \ldots + n_k = n}} \binom{n}{n_1, n_2, \ldots, n_k}n_1^{n_1-2}n_2^{n_2-2}\ldots n_k^{n_k-2},$$
men det finns ett problem med hur vi räknat hittills -- vi har pratat om ``det första trädet'', ``det andra trädet'', och så vidare, men träden skall ju inte ha etiketter, bara de enskilda noderna. Så vi behöver dividera med antalet sätt att sätta etiketter på träden, vilket är $k!$, och då får vi exakt den formel uppgiften bad oss om.

\section{Fråga 1, del b}

Vi börjar med att bekräfta att randvillkoren gäller. Att $t_{n,n}$ är $1$ för alla $n$ är enkelt att se, eftersom det ju bara finns en enda skog med lika många träd som noder -- det är grafen utan kanter på $n$ noder. Att $t_{n,0} = 0$ följer av att varje nod måste tillhöra något träd, så vi kan inte ha noll träd. Att $t_{n,k} = 0$ om $n < k$ följer av att varje träd måste ha åtminstone en nod, så vi kan inte ha fler träd än noder.

Låt oss nu bevisa rekursionen: Antag att vi vill konstruera en skog på $n$ noder och $k$ träd. Vi kan antingen göra detta genom att ta en skog på lika många träd men en nod färre, och sedan lägga till vår nya nod till ett av träden, eller ta en skog på en nod och ett träd färre, och lägga till vår nya nod som ett nytt träd på en enda nod.

I det tidigare fallet kommer vår nya nod ha etikett $n$, alltså högre än varje nod som redan var i trädet -- så vi måste lägga till den som ett löv, eftersom etiketteringen annars inte vore ökande. Men vi kan lägga till den som barn till vilken som helst av de $n-1$ redan existerande noderna.

I det senare fallet kan vi uppenbarligen bara göra på ett enda sätt -- och ett träd på en enda nod har trivialt en ökande etikettering.

Alltså är det totala antalet sätt att skapa vår skog
$$(n-1)t_{n-1,k} + t_{n-1,k-1}$$
och vi har bevisat rekursionen.
%\bibliography{references}
%\bibliographystyle{plainnat}

\end{document}
