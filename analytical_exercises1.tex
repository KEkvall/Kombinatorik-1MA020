\documentclass[nobib]{tufte-handout}

\title{Övningstillfälle 1: Teoretiska övningar $\cdot$ 1MA020}

\author[Vilhelm Agdur]{Vilhelm Agdur\thanks{\href{mailto:vilhelm.agdur@math.uu.se}{\nolinkurl{vilhelm.agdur@math.uu.se}}}}

\date{24 januari 2023}


%\geometry{showframe} % display margins for debugging page layout

\usepackage{graphicx} % allow embedded images
  \setkeys{Gin}{width=\linewidth,totalheight=\textheight,keepaspectratio}
  \graphicspath{{graphics/}} % set of paths to search for images
\usepackage{amsmath}  % extended mathematics
\usepackage{booktabs} % book-quality tables
\usepackage{units}    % non-stacked fractions and better unit spacing
\usepackage{multicol} % multiple column layout facilities
\usepackage{lipsum}   % filler text
\usepackage{fancyvrb} % extended verbatim environments
  \fvset{fontsize=\normalsize}% default font size for fancy-verbatim environments

\usepackage{color,soul} % Highlights for text


\include{mathcommands.extratex}

\begin{document}

\maketitle% this prints the handout title, author, and date

\begin{abstract}
\noindent
Detta dokument innehåller en samling övningar i kombinatorik som inte kräver någon programmering, utan är avsedda att lösas med papper och penna. Merparten av dessa problem är svårare än potentiella tentaproblem, eftersom de är avsedda att lösas i grupp över en längre tid, inte snabbt och individuellt i en tentasal.
\end{abstract}

\begin{xca}
    Det finns $k$ stycken olika sorters vykort i en butik, och du vill skicka ett vykort till varje av dina $n$ vänner.

    \begin{enumerate}
        \item Hur många sätt kan du göra detta på om det inte finns några ytterligare begränsningar?
        \item Vad blir svaret om alla korten du skickar skall vara olika?
        \item Vad blir svaret om du skall skicka två olika vykort till varje vän, men olika vänner kan få samma vykort?
    \end{enumerate}
\end{xca}

\begin{xca}
    Bevisa att $\binom{n}{k}$ och $\stirlingPart{n}{k}$ är polynom i $n$ om vi håller $k$ fixt.
\end{xca}

\begin{xca}
    Ge ett kombinatoriskt bevis för följande likhet, där vi antar att $x$ är ett heltal:
    $$\sum_{k = 0}^{n} \stirlingPart{n}{k}x(x-1)\ldots(x-k+1) = x^n.$$
\end{xca}

\begin{xca}
    Låt $B_n$ beteckna det $n$te \emph{Bell-talet}, vilket ger antalet mängdpartitioner av en mängd av $n$ element oavsett antal delar.\sidenote[][]{Alltså är $$B_n = \sum_{k = 1}^{n} \stirlingPart{n}{k}.$$}

    \begin{enumerate}[label = \roman*)]
        \item Finn en rekursion för Bell-talen.
        \item Bevisa att\sidenote[][]{Det är inte nödvändigtvis så att den enklaste lösningen använder rekursionen ni just funnit, även om ordningen på övningarna så klart antyder det.}
        $$B_n = \frac{1}{e}\sum_{k = 0}^{\infty} \frac{k^n}{k!}.$$
        \item Ge ett kombinatoriskt bevis för att
        $$B_n = \sum_{\substack{k_1, k_2, \ldots, k_n \geq 0\\k_1 + 2k_2 + \ldots nk_n = n}} \frac{n!}{\prod_{i=1}^{n} k_i!(i!)^{k_i}}.$$
    \end{enumerate}
\end{xca}

\begin{xca}
    Bevisa att
    $$\sum_{k = 0}^{\floor{\frac{n}{2}}} \binom{n}{2k} = 2^{n-1},$$
    med ett kombinatoriskt och ett algebraiskt bevis.\sidenote[][]{Ledtråd för det algebraiska: Betrakta också $\sum_{k = 0}^{\floor{\frac{n}{2}}} \binom{n}{2k}$. Vad blir dessa två summor tillsammans? Kan du använda det för att få en formel för uttrycket?}
\end{xca}

\begin{xca}
    Antag att ett heltal $n$ har primtalsfaktorisering $n = p_1^{\alpha_1}p_2^{\alpha_2}\ldots p_r^{\alpha_r}$. Använd inklusion-exklusion för att ge en formel för antalet heltal mellan $1$ och $n$ som är relativt prima till $n$, det vill säga antalet $j \in [n]$ sådana att den största gemensamma delaren till $j$ och $n$ är $1$.
\end{xca}

%\bibliography{references}
%\bibliographystyle{plainnat}

\end{document}
